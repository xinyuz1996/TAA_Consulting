\section{Drew's Questions}
\begin{itemize}
    \item Are you interested in answering the same questions as before? Is your primary goal to determine whether there is a relationship between TAA routinization and the 5 independent variables and the direction of that relationship?
    \item If you are interested in the same questions, do you want to better justify the regression analysis that was used in the paper or do you want to explore alternative methods for answering these questions?
    \item The reviewer expressed some concern about missing data. The missing values were among the independent variables is that right? Did respondents give any reason for why they left these items blank?
    \item Are you wanting to answer other questions with the data?
    \item Most of the data was collected using a 5 item Likert scale. The variables used in the analysis are combinations of these 5 item responses. How were the dependent and independent variables in the analysis constructed from these original survey responses?
    \item What was the goal of the principal components analysis you conducted?
\end{itemize}


\section{Xinyu's Questions}

\begin{itemize}
    \item Would you mind introducing the files in the google drive for the next meeting?
    \item Which language are we going to use for the future analysis, SPSS, SAS or R?
    \item As mentioned in a review comment, we can see that there are missing values ased on Panel A table 1. How do you treat the missing values before?
    \item We have a lot of questions about how the analysis was conducted before, and it would be best to have some previous codes to work with.
\end{itemize}

\section{Qiang's Questions}
\begin{itemize}
\item Not much explaination was given regarding how the variables in the paper stem from the original survey responses. We are going to need more details.
\item There was some missing data in the survey responses, how was it handled?
\item Was there any information in the survey deemed redundant and not utilized?
\item Was the 5 item Likert scales a common practice in accounting research or was it a subjective choice?
\end{itemize}



\section{Letter}

Hi Dr. Chen,

Before any analysis of the data can take place, we will most likely need to transform the survey data so that we are dealing with three response variables (initiation, adoption, and routinization) and so that we are dealing with 5 independent variables instead of one independent variable for each survey question. 

Since we may not have an opportunity to talk with you collaborator who was initially responsible for transforming the data, we will have to come up with a new scheme for this ourselves. In order to create new variables from the survey data that are meaningful for the analysis, it is critical that we have your input or the input of your other collaborators. We can perform variable transformation by taking the mean of the different survey question results, but that may not be meaningful for this application. We have to start by answering the question: what is a meaningful way of combining the results from different survey questions to obtain variables for our analysis? We need your input to answer this question in a satisfactory way.

Please know that this data transformation process may take some time depending on how we do it. It is also worth noting that using mean imputation for missing data may not be the most appropriate way of handling missing data in this case. It is simple and quick, and we can use this method, but we must add the caveat that it may bias the results inappropriately. Once the data transformation is complete and we are satisfied we have a set of variables that make sense for performing analysis, we can proceed with constructing models to examine the relationship between the constructed independent variables and the initiation, adoption, and routinization response variables. 

Best,

Drew, Xinyu, and Qiang




\section{Questions Towards Data (09/21/2020)}




After going over the questionnaire, we have summarized the following questions to address:

\begin{itemize}
    \item For Question 6, is it just a count of how many technologies have been adopted? Would it be worth considering a weighting for each of the technology categories?
    
    A: Weights: PRA $<$ Data mining $<$ AI % - reference
    
    \item For Q40 which is an open question related to the TAA title whose corresponding constructed variable is a binary variable asking whether the title is a TAA title, would you please tell us more about how you performed imputation for the missing value for this binary variable?
    
    \item For Q22, what does "Dichotomous: 6+7+8+9+10=1" in column BI  mean, and why is level "9" ignored in the questionnaire?
    
    \item For question 24 related to the global size, is it best to treat this as a four level categorical variable or a binary variable indicating whether it's domestic or international? 
    
    \item For the foreign sales and foreign procurement in Q29 and Q41, are you open to using the original continuous numerical value?
    
    \item For Q30, related to managerial obstacles, and other Likert scale questions, do we need to weight some survey items as more important than others in forming a composite score? 
    
\end{itemize} 




\section{Final Question List}
\begin{itemize}
    \item Could you introduce us to the file structure of the google drive? Which file is used to transform the survey data to allow for analysis? Which files (SPSS files) are used for the analysis?
    \item What procedure is used to combine the likert responses to create variables for regression? Is it common to use likert scales in accounting research?
    \item Was there any information collected on the surveys that was not included in the analysis?
    \item What procedure did you use for handling missing data?
    \item What issues from the reviewer's comments do you want us to address? The missing data issue, the small sample size issue? Both? We want to make sure we are very clear on what problem it is that you want us to solve. Do you want us to provide better justification for the regression analysis?
\end{itemize}