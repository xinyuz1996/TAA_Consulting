\documentclass[12pt, letterpaper]{article}
\usepackage[margin=1.5in]{geometry}
\usepackage{graphicx}
\usepackage{caption}
\usepackage{subfigure}
\usepackage[utf8]{inputenc}
\usepackage[english]{babel}
\usepackage{amsthm}
\usepackage{amssymb}
\usepackage{amsmath}
\usepackage{amsfonts}
\usepackage{xcolor}
\usepackage{float}
\begin{document}
\title{Tax Analytics and Automation Diffusion in Large Companies}
\date{September 12, 2020}
\author{Drew Hollis, Xinyu Zhang, and Qiang Heng NCSU Department of Statistics \\with Dr. Al Chen, Department of Accounting, Poole School of Management}
\maketitle

For this project, we cooperate with Dr. Al Chen in the Department of Accounting to conduct research related to Tax Analytics and Automation Diffusion in Large Companies. In order to stay relevant and competitive, businesses must take advantage of the latest technologies. Many new technologies and tools can automate mundane business administration tasks. Automating administrative tasks leads to greater efficiency and fewer errors and it frees up employees to focus on tasks that add more value to the company. All of these principles hold true for a company's tax and accounting department as well. \\

The advent of robotic process automation (RPA), the automation of business processes using artificial intelligence and software, has had a particularly significant impact on the accounting profession. \cite{mezzio2019robotic} The extent to which companies and accounting departments have adopted and integrated automated tax tools into their workplace varies. The aim of our client's study is to determine what factors seem to have the most significant impact on the initiation, adoption, and routinization of tax analytics and automation (TAA) tools in larger companies. \\

Initiation, adoption, and routinization are explained in \cite{cooper1990information}.  Initiation refers to an initial evaluation of the suitability of a technology by the company, adoption refers to the the point at which a company recognizes a technology as valuable for its interests and begins to integrate that technology into its business practices, and routinization refers to the full-scale deployment of that technology. \\

In determining what factors most impact the initiation, adoption, and routinization of a technology by a company, our client employed the technology-organization-environment (TOE) framework that was first proposed in \cite{tornatzky1990process} and later used by Zhu, et al. \cite{zhu2006process} to study the diffusion of e-business technology in companies.\\

The TOE framework indicates that technology diffusion is impacted by the technological sophistication and preparedness of a company, the organizational and managerial attributes of a company, and the the regulatory and competitive context or environment of a company. \\

Our client is interested in testing five hypotheses regarding TAA tools using the TOE framework:

\begin{enumerate}
\item Technology readiness is positively related to TAA initiation/adoption/routinization.
\item Technology integration is positively related to TAA initiation/adoption/routinization.
\item Managerial obstacles are negatively related to TAA initiation/adoption/routinization.
\item Competition intensity is positively related to TAA initiation/adoption/routinization.
\item A supportive regulatory environment is positively related to TAA initiation/adoption/routinization.
\end{enumerate}

To answer these questions, our client sent surveys to a chief tax officer at each of the Fortune 1000 companies asking about the intiation, adoption and routinization of TAA within that company's tax department and the level of technological readiness, technological integration, managerial obstacles, competition intensity, and regulatory support for TAA for the company.\\

Several of the variables like initiation, routinization, managerial obstacles, technology integration, competition intensity, and regulatory environment were measured using a number of Likert scale survey items. For instance to measure initiation, respondents were asked to use a 5-point Likert scale on 7 survey items to rate the significance that each of 7 potential TAA benefits had in their initial decision to pursue TAA tools. Similarly, technology integration was measured using a 5-point Likert scale with two survey items, one asking how integrated a company's tax systems are with their own information systems and one asking how integrated their tax systems are with business partner systems. \\

Other variables like technological readiness where measured by asking a series of questions relating to the kinds of technology used at the company and the number of tax employees at the company who specialize in the use of TAA tools. \\

Of the 1000 companies that received surveys, 70 responded. The client has already performed an analysis answering the hypotheses for the routinization phase of the technology diffusion process. He wants us to perform analysis that explores the intitation, adoption, and routinization phases simultaneously. 



\end{document}